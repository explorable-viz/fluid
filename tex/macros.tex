% Misc.
\newcommand{\ulparagraph}[1]{\paragraph{\textnormal{\ul{#1}}}}
\newcommand{\SeeSupplementaryVersion}{See companion technical report.}
\newcommand{\Set}[1]{\textsf{#1}}
\renewcommand{\separated}[2]{#1\downrsquigarrow#2}
\newcommand{\precedes}{\preccurlyeq}
\newcommand{\support}[1]{|#1|}
\newcommand{\vertices}[1]{V_{#1}}
\newcommand{\edges}[1]{E_{#1}}
\newcommand{\labelfun}[1]{\phi_{#1}}
\newcommand{\graphof}[1]{|#1|}
\newcommand{\partialto}{\nrightarrow}
\newcommand{\intersection}{\cap}
\newcommand{\union}{\cup}

% Categories.
\newcommand{\cat}[1]{\mathbf{#1}}
\newcommand{\homset}[3]{#1_{#2,#3}}
\newcommand{\op}[1]{#1^\textsf{Op}}

% Specific categories and sets.
\newcommand{\Conf}{\cat{Conf}}

% Symbols.
\newcommand{\eqifdef}{\simeq}
\newcommand{\binderSymbol}{\mathrel{\rlap{$\leftspoon$}\hspace*{0.2em}\rightfilledspoon}}
\newcommand{\id}[1]{\textsf{id}_{#1}}

% Relational composition.
\DeclareSymbolFont{bbsymbol}{U}{bbold}{m}{n}
\DeclareMathSymbol{\bbsemi}{\mathbin}{bbsymbol}{"3B}

% Hacked from StackOverflow.
\providecommand{\eqeq}{
  \mathrel{% it's a relation
  \smash{% we don't want that it influences the interline spacing
  \vcenter{% the symbol will be vertically centered
    \offinterlineskip % no interline skip here
    \ialign{% build a table
       \hfil##\hfil\cr % just one centered column
       $=$\cr % first row
       \noalign{\kern-.0ex}% shorten the vertical distance
       $=$\cr % second row
    }% end of the \ialign
  }% end of \vcenter
  }% end of \smash
  \vphantom{=}% pretend it's as high as a =
  }% end of \mathrel
}

% Setup the matha font (from mathabx.sty) -- because mathabx.sty incompatible with
% amsmath package.
\DeclareFontFamily{U}{matha}{\hyphenchar\font45}
\DeclareFontShape{U}{matha}{m}{n}{
      <5> <6> <7> <8> <9> <10> gen * matha
      <10.95> matha10 <12> <14.4> <17.28> <20.74> <24.88> matha12
      }{}
\DeclareSymbolFont{matha}{U}{matha}{m}{n}

% Define some characters from mathabx.dcl.
\DeclareMathSymbol{\second}{3}{matha}{"32}
\DeclareMathSymbol{\third}{3}{matha}{"33}
\DeclareMathSymbol{\fourth}{3}{matha}{"34}

% Adjoints
\newcommand{\adjointSymbol}{*}
\newcommand{\loweradj}[1]{{#1}_{\adjointSymbol}}
\newcommand{\upperadj}[1]{{#1}^{\adjointSymbol}}
\newcommand{\reverse}[2]{{#1}_{\leftarrow#2}}

% Dotted frac
\makeatletter
\newcommand{\dotfrac}[2]{
\mathchoice
{\ooalign{$\genfrac{}{}{0pt}{0}{#1}{#2}$\cr\leavevmode\cleaders\hb@xt@ .22em{\hss $\displaystyle\cdot$\hss}\hfill\kern\z@\cr}}
{\ooalign{$\genfrac{}{}{0pt}{1}{#1}{#2}$\cr\leavevmode\cleaders\hb@xt@ .22em{\hss $\textstyle\cdot$\hss}\hfill\kern\z@\cr}}
{\ooalign{$\genfrac{}{}{0pt}{2}{#1}{#2}$\cr\leavevmode\cleaders\hb@xt@ .22em{\hss $\scriptstyle\cdot$\hss}\hfill\kern\z@\cr}}
{\ooalign{$\genfrac{}{}{0pt}{3}{#1}{#2}$\cr\leavevmode\cleaders\hb@xt@ .22em{\hss $\scriptscriptstyle\cdot$\hss}\hfill\kern\z@\cr}}
}
\makeatother

% Configurations.
\newcommand{\lbrackActive}{\llbracket}
\newcommand{\rbrackActive}{\rrbracket}
\newcommand{\lbrackRewind}{\llangle}
\newcommand{\rbrackRewind}{\rrangle}
\newcommand{\tensorUnit}{I}
\newcommand{\simult}[2]{\dotfrac{#1}{#2}}
\newcommand{\apply}[2]{#1[#2]}
\newcommand{\applySimult}[3]{#1\left[\simult{#2}{#3}\right]}
\newcommand{\applyBibble}[2]{#1\left[#2\right]}
\newcommand{\applyActive}[2]{#1\textcolor{blue}{\lbrackActive}#2\textcolor{blue}{\rbrackActive}}
\newcommand{\applyRewind}[2]{#1\textcolor{red}{\lbrackRewind}#2\textcolor{red}{\rbrackRewind}}
\newcommand{\inputs}[1]{\textsf{in}(#1)}
\newcommand{\outputs}[1]{\textsf{out}(#1)}
\newcommand{\comp}{\mathrel{\bbsemi}}

% Self-explaining computation.
\newcommand{\SEV}[2]{#1:#2}
\newcommand{\emptyE}{\cdot}

% Reference semantics.
\newcommand{\Ref}[1]{#1}
\newcommand{\evalSymbol}{\Rightarrow}
\newcommand{\unevalSymbol}{\Leftarrow}
\newcommand{\eval}{\Ref{\evalSymbol}}
\newcommand{\evalF}[1]{\operatorname{\textsf{eval}}_{#1}}
\newcommand{\uneval}{\unevalSymbol}
\newcommand{\unevalF}[1]{\operatorname{\textsf{uneval}}_{#1}}
\newcommand{\reduceF}[1]{\textsf{reduce}_{#1}}
\newcommand{\unreduceF}[1]{\textsf{unreduce}_{#1}}
\newcommand{\stepF}[1]{\textsf{step}_{#1}}
\newcommand{\unstepF}[1]{\adjoint{\textsf{step}}_{#1}}
\newcommand{\trUnreduceF}[1]{\candidate{\textsf{unreduce}}_{#1}}
\newcommand{\trUnstepF}[1]{\candidate{\textsf{unstep}}_{#1}}

% Tracing.
\newcommand{\reduceSymbol}{\twoheadrightarrow}
\newcommand{\traceSymbol}{\smallsquare}
\newcommand{\preTrace}[1]{\traceSymbol\hspace*{-0.4em}#1}
\newcommand{\postTrace}[1]{#1\hspace*{-0.4em}\traceSymbol}
\newcommand{\unreduceSymbol}{\twoheadleftarrow}
\newcommand{\stepSymbol}{\xrightarrow{\mathmakebox[0.55em]{}}}
\newcommand{\unstepSymbol}{\xleftarrow{\mathmakebox[0.55em]{}}}
\newcommand{\refreduce}{\mathrel{\reduceSymbol}}
\newcommand{\reduce}{\mathrel{\preTrace{\reduceSymbol}}}
\newcommand{\unreduce}{\mathrel{\postTrace{\unreduceSymbol}}}
\newcommand{\refunreduce}{\mathrel{\adjoint{\reduceSymbol}}}
\newcommand{\refstep}{\mathrel{\stepSymbol}}
\newcommand{\step}{\mathrel{\preTrace{\stepSymbol}}}
\newcommand{\unstep}{\mathrel{\postTrace{\unstepSymbol}}}
\newcommand{\trEval}{\traceSymbol\hspace*{-0.4em}\evalSymbol}
\newcommand{\redexTr}[2]{#1\vdash#2}

% Slicing.
\newcommand{\rewindSymbol}{\smalltriangleleft}
\newcommand{\postRewind}[1]{#1\hspace*{-0.4em}\rewindSymbol}
\newcommand{\rewinduneval}{\mathrel{\rewindSymbol\hspace*{-0.25em}\rewindSymbol}}
\newcommand{\rewindunevalstep}{\mathrel{\rewindSymbol}}
\newcommand{\rewind}{\mathrel{\rewindSymbol\hspace*{-0.25em}\rewindSymbol}}
\newcommand{\rewindF}[1]{\textsf{rewind}_{#1}}
\newcommand{\rewindstep}{\mathrel{\postRewind{\unstepSymbol}}}%{\mathrel{\rewindSymbol}}
\newcommand{\trUneval}{\unevalSymbol}
\newcommand{\trUnevalF}[1]{\textsf{tr-uneval}_{#1}}
\newcommand{\trBwdSlice}{\nwarrow}
\newcommand{\trBwdSliceF}[1]{\textsf{bwd-slice}_{#1}}
\newcommand{\trFwdSlice}{\nearrow}
\newcommand{\tensorProc}[2]{{#1}\rightsquigarrow{#2}}
\newcommand{\collapse}[1]{\ulcorner #1 \urcorner}
\newcommand{\sqgeq}{\sqsupseteq}
\newcommand{\sqleq}{\sqsubseteq}
\newcommand{\sqlt}{\sqsubset}
\newcommand{\cxtsqleq}[1]{\sqleq_{#1}}
\newcommand{\up}[2]{#1\uparrow#2}
\newcommand{\down}[1]{{\mathrel{\raisebox{0.08em}{$\downarrow$}}}#1}
\newcommand{\leastConfig}[1]{{\bot_{#1}}}
\newcommand{\leastTrace}[1]{{\bot_{#1}}}
\newcommand{\compUnit}[1]{{\textsf{id}_{#1}}}
\newcommand{\holeId}[1]{{\prHole^{#1}}}

% Recursive definitions
\newcommand{\defsEmpty}{\varnothing}
\newcommand{\defRec}[3]{{#1}{#2}.{#3}}

% Substitution.
\newcommand{\subst}[3]{{#1}\{{#2}/{#3}\}}
\newcommand{\unsubst}[3]{\{{#2}/{#3}\}{#1}}
\newcommand{\substF}[1]{\textsf{subst}_{#1}}
\newcommand{\unsubstF}[1]{\textsf{unsubst}_{#1}}
\newcommand{\recSubstF}[1]{\textsf{subst}_{#1}}
\newcommand{\recUnsubstF}[1]{\textsf{unsubst}_{#1}}
\newcommand{\unsubstVal}[2]{\phi_{#1}(#2)}
\newcommand{\unsubstExp}[2]{\theta_{#1}(#2)}

% Patterns.
\newcommand{\match}[2]{#1.#2}
\newcommand{\matches}[2]{\vec{\match{#1}{#2}}}
\newcommand{\matchval}[3]{\mathsf{match}\;#1\;\mathsf{as}\;\matches{#2}{#3}}

% Expressions and traces.
\newcommand{\exBody}[3]{{#1}{#2}.{#3}}
\newcommand{\exApp}[2]{{#1}\;{#2}}
\newcommand{\exAppLive}[3]{\exApp{#1}{#2}.#3}
\newcommand{\exClos}[2]{\langle{#1},{#2}\rangle}
\newcommand{\exFst}[1]{\kw{fst}\;{#1}}
\newcommand{\exInl}[1]{\kw{inl}\;{#1}}
\newcommand{\exInr}[1]{\kw{inr}\;{#1}}
\newcommand{\exLetrec}[2]{\kw{letrec}\;{#1}\;\kw{in}\;{#2}}
\newcommand{\exMatchAs}[2]{\kw{match}\;#1\;\kw{as}\;#2}
\newcommand{\exMatchedAs}[2]{\kw{match}\;#1.#2}
\newcommand{\exPair}[2]{(#1,#2)}
\newcommand{\exPatternLambda}[2]{\lambda\matches{#1}{#2}}
\newcommand{\exPrimConst}[1]{#1}
\newcommand{\exPrimOp}{\primOpSymbol}
\newcommand{\exRec}[2]{\langle{#1},{#2}\rangle}
\newcommand{\exSnd}[1]{\kw{snd}\;{#1}}
\newcommand{\exUnit}{()}
\newcommand{\exVal}[2]{{#1}_{#2}}

% Prim ops.
\newcommand{\primOp}{\mathbin{\hat{\primOpSymbol}}}
\newcommand{\primOpAdjoint}[2]{\mathbin{\primOp_{#1,#2}^{-1}}}
\newcommand{\primOpSymbol}{\oplus}
\newcommand{\primOpApp}[2]{{#1}\primOp{#2}}

\makeatletter
\newcommand{\superimpose}[2]{%
  {\ooalign{$#1\@firstoftwo#2$\cr\hfil$#1\@secondoftwo#2$\hfil\cr}}}
\makeatother

% Sequences.
\renewcommand{\vec}[1]{\overrightarrow{#1}}
\newcommand{\seq}[1]{{#1}\hspace*{-0.1em}^{\thinstar}}
\newcommand{\seqEmpty}{{\small\bullet}}

% Theorems, definitions, etc.
\newtheorem{Theorem}{Theorem}
\newtheorem{Corollary}{Corollary}
\newtheorem{Lemma}{Lemma}
\newtheorem{definition}{Definition}
\newtheorem{Example}{Example}

\definecolor{shadecolor}{rgb}{0.9,0.9,0.9}
\newenvironment{lemma}{\begin{snugshade}\begin{Lemma}}{\end{Lemma}\end{snugshade}}
\newenvironment{theorem}{\begin{snugshade}\begin{Theorem}}{\end{Theorem}\end{snugshade}}
\newenvironment{corollary}{\begin{snugshade}\begin{Corollary}}{\end{Corollary}\end{snugshade}}
\newenvironment{example}{\begin{snugshade}\begin{Example}\small}{\end{Example}\end{snugshade}}

% Comments
\newcommand{\notes}[1]{\textcolor{red}{#1}}
\newcommand{\dg}[1]{\notes{Deepak says: #1}}
\newcommand{\rp}[1]{\notes{Roly says: #1}}
